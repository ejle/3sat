% !TEX root = survey.tex

\documentclass[12pt, a4paper]{article}
\usepackage{amsmath}
\usepackage{amssymb}
\usepackage{amsthm}
\usepackage{epsfig}
\usepackage[margin=1in]{geometry}
%\usepackage{enumitem}
\usepackage{enumerate}
\usepackage{graphicx}
\usepackage{tikz}
\usetikzlibrary{automata, positioning}
\usepackage{caption}
\usepackage{mathtools}
\usepackage{algorithm}
\usepackage{algorithmic}
\usepackage{todonotes}
\usepackage{times}
\usepackage{pdfpages}
\usepackage{mathtools}
%\usepackage[]{algorithm2e}

% For writing code:
\usepackage{fancyvrb}
\DefineVerbatimEnvironment{code}{Verbatim}{fontsize=\small}
    \def\showvrb#1{%
    \texttt{\detokenize{#1}}%
}

\linespread{1.1}
\setlength{\parindent}{0pt} % Sets the indent length to 0
\setlength{\parskip}{8pt plus 1pt minus 1pt} % paragraph vertical distance

\everymath{\displaystyle} % displays inlinlatexe math as displaymath


\title{3SAT Survey}
\author{Edward Lee}

%%%%%%%%%%%%%%%%%%%%%%%%%%
% Commonly used theorems %
%%%%%%%%%%%%%%%%%%%%%%%%%%

% Use roman for text - numbering follows onwards
\theoremstyle{definition}
\newtheorem{defn}{Definition}[section]
\newtheorem{them}{Theorem}[section]
\newtheorem{lem}{Lemma}[them]
\newtheorem{prop}[them]{Proposition}
\newtheorem{corr}[them]{Clat\exlatexorollatexlary}
\newtheorem{corrr}{Corollary}[them] %for consistency issues
% examples follatexlow different numbering
\newtheorem{eg}{Example}[section]
\newtheorem{egg}[them]{Example}
\newtheorem*{themf}{Theorem 5.0}
\newtheorem*{themm}{Theorem}
\newtheorem{question}{Question}

%%%%%%%%%%%%%
% Shortcuts %
%%%%%%%%%%%%%

\newcommand{\Q}{\mathbb{Q}}
\newcommand{\C}{\mathbb{C}}
\newcommand{\F}{\mathbb{F}}
\newcommand{\R}{\mathbb{R}}
\newcommand{\Z}{\mathbb{Z}}
\newcommand{\N}{\mathbb{N}}
\newcommand{\D}{\mathbb{D}}

\newcommand{\mcC}{\mathcal{C}}
\newcommand{\mcN}{\mathcal{N}}
\newcommand{\mcE}{\mathcal{E}}
\newcommand{\mcP}{\mathcal{P}}
\newcommand{\mcF}{\mathcal{F}}
\newcommand{\mcS}{\mathcal{S}}
\newcommand{\mcA}{\mathcal{A}}
\newcommand{\mcB}{\mathcal{B}}
\newcommand{\mcX}{\mathcal{X}}
\newcommand{\mcL}{\mathcal{L}}
\newcommand{\mcH}{\mathcal{H}}
\newcommand{\mcl}{\mathcal{l}}

\newcommand{\bfe}{\mathbf{e}}
\newcommand{\mbf}{\mathbf}

\newcommand{\ran}{\mbox{Ran}}
 
\newcommand{\al}{\alpha}
\newcommand{\s}{\sigma}
\newcommand{\ep}{\epsilon}
\newcommand{\dl}{\delta}
\newcommand{\sg}{\sigma}

\newcommand{\bs}{\backslash}
\newcommand{\n}{\\}
\newcommand{\ol}{\overline}

\newcommand{\ltlU}{\textsf{ \bfseries UNTIL }}
\newcommand{\ltlF}{\textsf{\bfseries F }}
\newcommand{\ltlG}{\textsf{\bfseries G }}
\newcommand{\ltlX}{\textsf{\bfseries X }}
\newcommand{\ltlA}{\textsf{\bfseries A }}
\newcommand{\ltlE}{\textsf{\bfseries E }}

%%%%%%%%%%%%%%%%%%%%%%%%%
% Standard big brackets %
%%%%%%%%%%%%%%%%%%%%%%%%%

\newcommand{\floor}[1]{\left\lfloor{#1}\right\rfloor}
\newcommand{\ceil}[1]{\left\lceil{#1}\right\rceil}
\newcommand{\bbA}[1]{\left({#1}\right)}
\newcommand{\bbB}[1]{\left[{#1}\right]}
\newcommand{\bbC}[1]{\left\{{#1}\right\}}

%%%%%%%%%%%%%%%%%%
% Start Document %
%%%%%%%%%%%%%%%%%%

\begin{document}
\maketitle

\section{Sch\"{o}ning's Algorithm}


The paper \cite{SchoningA99} presents a simple randomized algorithm for solving $k$-SAT

\begin{itemize}
	\item Guess an initial assignment $a \in \{0,1\}^n$ uniformly at random
	\item Repeat $3n$ times: 
	\begin{itemize}
		\item If the formula is satisfied by $a$ - stop and accept $a$.
		\item Let $C$ be some clause not being satisfied by $a$: Pick one of the $\leq k$ literals in the clause at random and flip its value in the current assignment.
	\end{itemize}
\end{itemize}

We now wish to estimate the running time of this randomized algorithm, and upper bound it as $p$. The rationale behind this is that the expected number of repetitions before a satisfying assignment is $\frac{1}{p}$, and the probability that after $t$ repeitions, the probability of not getting a satisfying assignment is at most $(1-p)^t \leq e^{pt}$.

We provide analysis for 3 sat for now. General arguments can be found in the paper but this is a nicer simplification for presentation and understanding for now. First, there are $2^n \binom{n}{u}$ possible assignments of $a$ which differ from $a^*$ in exactly $u$ variables.

Next we consider the question of finding a solution from assignment $a$ in exactly $3u$ steps. 
% Let $a$ be the current assignment and $a^*$ be the satisfying assignment (for now just assume there is a unique solution since the analysis is similar for non-unique as well). 
A good step is one which flips a variable in which the assignment between $a$ and $a^*$ differ. A bad step is one which flips a variable in which $a$ and $a^*$ are the same. This means that we need to perform exactly $2u$ good steps and $u$ bad steps. The probability this occurs is
\[
	\binom{3u}{2u} \left(\frac{1}{3}\right)^{2u} \left( \frac{2}{3}\right)^{u}
\]
We then apply Stirling's formula:
\[
	\binom{3u}{2u} = \binom{3u}{u} \geq \frac{1}{\sqrt{5u}} \frac{3^{3u}}{2^{2u}}
\]
to obtain a nicer lower bound. We can hence put this together in order to calculate $p$, the expected probability of finding a satsifying assignments:

\begin{align}
	p 	&= \sum_{u = 0}^{n} 2^{-n} \binom{n}{u} \binom{3u}{2u} \left( \frac{1}{3} \right)^{2u} \left( \frac{2}{3} \right)^u \\
	 	&\geq \frac{1}{\sqrt{5n}} 2^{-n}\sum_{u = 0}^{n} \binom{n}{u} \frac{3^{3u}}{2^{2u}} \left( \frac{1}{3} \right)^{2u} \left( \frac{2}{3} \right)^u \\
	 	&\geq \frac{1}{\sqrt{5n}} 2^{-n}\sum_{u = 0}^{n} \binom{n}{u} \frac{1}{2^u} \\
	 	&\geq \frac{1}{\sqrt{5n}} \left(\frac{1}{2} \cdot \left(1 + \frac{1}{2} \right) \right)^n \\
	 	& \frac{1}{\sqrt{5n}} \cdot \left( \frac{3}{4} \right)^n
\end{align}

As the running time is just a constant factor of $\frac{1}{p}$ then the running time is $O\left(\frac{4}{3} \right)^n$.
	




\section{PPSZ Algorithm}

\cite{PaturiA98}




\bibliographystyle{plain}
\bibliography{3sat}

\end{document}













